\documentclass[../../thesis.tex]{subfiles}


\begin{document}
\begin{figure}[ht]
    \centering
    \tikzsetnextfilename{snells_law}
    \begin{tikzpicture}
        \tikzstyle arrowstyle=[scale=2]
        \tikzstyle directed=[postaction={decorate,decoration={markings,
            mark=at position .65 with {\arrow[arrowstyle]{stealth}}}}]
        \pgfdeclarelayer{bg}    % declare background layer
        \pgfdeclarelayer{bbg}    % declare backbackground layer
        \pgfsetlayers{bg,main}  % set the order of the layers (main is the standard layer)
        \fill[blue!10] (5,0) -- (5,6) -- (10,6) -- (10,0) -- cycle;
        \draw[dashed] (0,2) -- (10,2);
        \draw (5,0) -- (5,6);
        \draw[color=red] (-0.1,6 - 0.1) -- (0.1,6 + 0.1);
        \draw[color=red] (+0.1,6 - 0.1) -- (-0.1,6 + 0.1);
        \node[color = red] at (0.25,6.25) {A};
        \draw[color=red] (5-0.1,2 - 0.1) -- (5+0.1,2 + 0.1);
        \draw[color=red] (5+0.1,2 - 0.1) -- (5-0.1,2 + 0.1);
        \node[color = red] at (5.25,2.25) {B};
        \draw[color=red] (10-0.1,- 0.1) -- (10+0.1,+ 0.1);
        \draw[color=red] (10+0.1,- 0.1) -- (10-0.1,+ 0.1);
        \node[color = red] at (9.8,0.3) {C};
        \draw[color=red, directed] (0,6) -- (5,2);
        \draw[color=red, directed] (5,2) -- (10,0);
        \coordinate (a) at (0,6);
        \coordinate (b) at (5,2);
        \coordinate (c) at (0,2);
        \pic[draw," $ \alpha $ ", draw=orange, <->, angle eccentricity=1.2, angle radius=1.5cm] { angle=a--b--c } ;
        \coordinate (a) at (10,0);
        \coordinate (b) at (5,2);
        \coordinate (c) at (10,2);
        \pic[draw," $ \beta $ ", draw=orange, <->, angle eccentricity=1.2, angle radius=1.5cm] { angle=a--b--c } ;
        \node at (4.5,5.5) {$n_1$};
        \node at (5.5,5.5) {$n_2$};
        \draw [decorate,decoration={brace,amplitude=10pt,mirror,raise=4pt},yshift=0pt] (10,0) -- (10,6) node [black,midway,xshift=0.8cm] {\footnotesize
        $l$};
        \draw [decorate,decoration={brace,amplitude=10pt},xshift=-4pt,yshift=0pt] (0,2) -- (0,6) node [black,midway,xshift=-14pt] {\footnotesize $x$};
        \draw [decorate,decoration={brace,mirror,amplitude=10pt},xshift=0pt,yshift=-4pt] (0,0) -- (5,0) node [black,midway,yshift=-14pt] {\footnotesize $a$};
        \draw [decorate,decoration={brace,mirror,amplitude=10pt},xshift=0pt,yshift=-4pt] (5,0) -- (10,0) node [black,midway,yshift=-14pt] {\footnotesize $b$};
    \end{tikzpicture}
    \caption{Visual derivation of \textsc{Snell}'s law. }
    \label{fig:snells-law}
\end{figure}


\end{document}
