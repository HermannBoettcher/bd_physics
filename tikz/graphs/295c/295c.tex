\documentclass[../../../thesis.tex]{subfiles}

\begin{document}
  \begin{figure}[tb]
		\centering
    \subfloat[]{
      \tikzsetnextfilename{295c}
      \begin{tikzpicture}[spy using outlines={ellipse, magnification=4, connect spies}]
          \def\PrelMin{.86}
  				\def\PrelMax{1}
  				\def\TransMin{1e-6}
  				\def\TransMax{1}
  				\def\LfMin{0}
  				\def\LfMax{2}
          %
          \begin{axis}[
            /tikz/line join=bevel,
            width=0.8*\textwidth,
            height=0.5*\textwidth,
            grid,
  					axis y line*=left,
            legend style={at={(0,0.5)}, legend columns=1, anchor=north west},
            every axis plot,
  					axis y line*=left,
  					line width = 1pt,
  					xmin = \PrelMin, xmax = \PrelMax,
  					ymin = \LfMin, ymax = \LfMax,
  					xlabel = {Relative pressure $P_\mathrm{rel}$},
  					ylabel = {Liquid fraction $LF$},
  					ytick = {0,0.25,0.50,0.75,1},
            ]
  					% Add plots
  					\addplot[mark=none, color=blue] table [x=Prel,y=liquid_fraction]{tikz/graphs/295c/295c_cond_2.txt};
  					\addlegendentry{$LF_\mathrm{cond}$}
  					\addplot[mark=none, color=blue!50] table [x=Prel,y=liquid_fraction]{tikz/graphs/295c/295c_evap_2.txt};
  					\addlegendentry{$LF_\mathrm{evap}$}
          \end{axis}
  				% transmission
  				\begin{axis}[
            /tikz/line join=bevel,
            width=0.8*\textwidth,
            height=0.5*\textwidth,
  					ymode=log,
  					axis y line*=right,
  					line width = 1pt,
  					xmin = \PrelMin, xmax = \PrelMax,
  					ymin = \TransMin, ymax = \TransMax,
  					ylabel = {Transmission $T$},
  					xtick = {1e-1,1e-2,1e-3},
  					ytick = {1,1e-1,1e-2,1e-3},
  					grid,
            legend style={at={(0,0.5)}, legend columns=1, anchor=south west},
  					]
  					% Add plots
  					\addplot[mark=none, color=red] table [x=Prel,y=transmission]{tikz/graphs/295c/295c_cond_2.txt};
  					\addlegendentry{$T_\mathrm{cond}$}
  					\addplot[mark=none, color=red!50] table [x=Prel,y=transmission]{tikz/graphs/295c/295c_evap_2.txt};
  					\addlegendentry{$T_\mathrm{evap}$}
  					\coordinate (spypoint1) at (axis cs:0.94,0.6);
    				\coordinate (magnifyglass1) at (axis cs:0.9,.1);
  				\end{axis}
          \spy[size=5cm, height=1.5cm] on (spypoint1) in node[fill=white] at (magnifyglass1);
      \end{tikzpicture}
      \label{fig:295c}
      }
      \\
      \subfloat[]{
        \tikzsetnextfilename{295c_cp_op}
        \begin{tikzpicture}
            \def\PrelMin{40}
    				\def\PrelMax{70}
    				\def\TransMin{1e-6}
    				\def\TransMax{1}
    				\def\LfMin{0}
    				\def\LfMax{2}
            %
            \begin{axis}[
              /tikz/line join=bevel,
              width=0.8*\textwidth,
              height=0.5*\textwidth,
              grid,
    					axis y line*=left,
              legend style={at={(0,0.5)}, legend columns=1, anchor=north west},
              every axis plot,
    					axis y line*=left,
    					line width = 1pt,
    					xmin = \PrelMin, xmax = \PrelMax,
    					ymin = \LfMin, ymax = \LfMax,
    					xlabel = {Pore diameter $d_\mathrm{pore}$ in $\si{\nano\meter}$},
    					ylabel = {Liquid fraction $LF$},
    					ytick = {0,0.25,0.50,0.75,1},
              ]
    					% Add plots
    					\addplot[mark=none, color=red] table [x=r_kelvin,y=liquid_fraction]{tikz/graphs/295c/295c_cp_cond_2.txt};
    					\addlegendentry{$LF_\mathrm{cond}^\mathrm{cp}$}
    					\addplot[mark=none, color=blue] table [x=r_kelvin,y=liquid_fraction]{tikz/graphs/295c/295c_cond_2.txt};
    					\addlegendentry{$LF_\mathrm{cond}^\mathrm{op}$}
    					\addplot[mark=none, color=magenta!50] table [x=r_kelvin,y=liquid_fraction]{tikz/graphs/295c/295c_cp_evap_2.txt};
    					\addlegendentry{$LF_\mathrm{evap}^\mathrm{cp/op}$}
      				\coordinate (roi) at (axis cs:53,.37);
            \end{axis}
            \draw (roi) ellipse (2.55cm and 1.15cm);
    				% transmission
    				\begin{axis}[
              /tikz/line join=bevel,
              width=0.8*\textwidth,
              height=0.5*\textwidth,
    					ymode=log,
    					axis y line*=right,
    					line width = 1pt,
    					xmin = \PrelMin, xmax = \PrelMax,
    					ymin = \TransMin, ymax = \TransMax,
    					ylabel = {Transmission $T$},
    					xtick = {1e-1,1e-2,1e-3},
    					ytick = {1,1e-1,1e-2,1e-3},
    					grid,
              legend style={at={(0,0.5)}, legend columns=1, anchor=south west},
    					]
    					% Add plots
    					\addplot[mark=none, color=red] table [x=r_kelvin,y=transmission]{tikz/graphs/295c/295c_cp_cond_2.txt};
    					\addlegendentry{$T_\mathrm{cond}^\mathrm{cp}$}
    					\addplot[mark=none, color=blue] table [x=r_kelvin,y=transmission]{tikz/graphs/295c/295c_cond_2.txt};
    					\addlegendentry{$T_\mathrm{cond}^\mathrm{op}$}
    					\addplot[mark=none, color=magenta!50] table [x=r_kelvin,y=transmission]{tikz/graphs/295c/295c_cp_evap_2.txt};
    					\addlegendentry{$T_\mathrm{evap}^\mathrm{cp/op}$}
    				\end{axis}
        \end{tikzpicture}
        \label{fig:295c-op-cp}
      }
      \caption{The isotherm of membrane 295c is shown in \protect\subref{fig:295c}. To analyse the two step condensation, the isotherm is plotted on a pore diameter scale using the \textsc{Kelvin} equation (\cref{sec:kelvin-equation}) once under the assumption of condensation at equilibrium pressure for closed pores (cp) and once under that of spinodal pressure for open pores (op) in \protect\subref{fig:295c-op-cp}. For interpretations please read \cref{sec:theory-and-defects}.}
      \label{fig:295c-analysis}
  \end{figure}
\end{document}
