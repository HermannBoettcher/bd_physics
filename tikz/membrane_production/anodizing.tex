\documentclass[../../thesis.tex]{subfiles}


\begin{document}

\begin{figure}[ht]
    \centering
    \tikzsetnextfilename{anodizing}
    \begin{tikzpicture}
        \pgfdeclarelayer{bg}    % declare background layer
        \pgfsetlayers{bg,main}  % set the order of the layers (main is the standard layer)
        \definecolor{copper}{rgb}{0.72, 0.45, 0.2}
        \def\holerad{0.3}
        \def\factor{4}
        \def\xradius{2}
        \def\yradius{2/\factor}
        \def\height{1.05cm}
        \def\xandy{2 and 2/\factor}
        \tikzset{
          pics/.cd, %
          disc/.style ={
            code = {
              %% the foundation
              \path [fill=black!15] (-\xradius,0) -- (-\xradius,-\height) arc
              (180:360:\xandy) -- (\xradius,0) arc (0:180:\xandy);%
              \path [top color=black!25, bottom color=white, opacity=0.2] (0,0) ellipse
              [x radius=\xradius, y radius =\yradius];%
              \path [left color=black!25, right color=black!15] (-\xradius,0) --
              (-\xradius,-\height) arc (180:240:\xandy) -- +(0,\height) arc
              (240:180:\xandy);%
              \path [left color=black!15, right color=black!30] (\xradius,0) --
              (\xradius,-\height) arc (360:320:\xandy) -- +(0,\height) arc
              (320:360:\xandy);
              %% rays in front
              \foreach \col/\r/\shift/\stop/\opacity in {%
                black/205/25/20/100, %
                black/295/35/30/100, %
                black/295/30/30/200, %
                black/295/25/20/300, %
                white/245/14/14/100, %
                white/245/12/12/20, %
                white/245/10/10/10} {%
                \foreach \i [evaluate={\opposite=\r-180;}] in {0,1,...,\stop}{%
                  \fill [\col, fill opacity = 1/\opacity] (\opposite:0.1 and
                  0.1/\factor) -- (\r+\shift-\i:\xandy) -- ++(0,-\height) arc
                  (\r+\shift-\i:\r-\shift+\i:\xandy) -- +(0,\height) -- cycle; }}
              %% rays in back
              \foreach \r/\shift/\stop/\opacity in {%
                25/25/20/100, %
                115/35/3/150,%
                115/30/23/100} {%
                \foreach \i [evaluate={\opposite=\r-180;}] in {0,1,...,\stop}{%
                  \fill [black, fill opacity = 1/\opacity] (\opposite:0.1 and 0.1/\factor) --
                  (\r+\shift-\i:\xandy) arc (\r+\shift-\i:\r-\shift+\i:\xandy) --
                  cycle; }}
              %% masking the four edges in the center
              \foreach \i in {0.1, 0.2, ..., 0.4}%
              \fill[black!15, opacity=0.7, path fading=fade out]
              (0,0) ellipse[x radius=\i, y radius =\i/\factor];
              %% the light and the dark arcs
              \foreach \i [evaluate={\start=185+10*\i; \finish=355-10*\i;}]%
              in {0.1, 0.2, ..., 1.5}{%
                \draw[white, opacity=0.04, line width=\i, yshift=0.02cm]
                (\start:\xandy) arc (\start:\finish:\xandy);
                \draw[black!80, opacity=0.05, line width=\i, yshift=-\height]
                (\start:\xandy) arc (\start:\finish:\xandy); }
            }
          },%
          disc bottom/.style = {
            code = {
              \foreach \i/\opacity in {%
                1/20,2/20,3/20,4/30,5/35,6/40,7/60,8/80,9/100,10/100,11/100,12/100}%
                \fill [red, fill opacity = 1/\opacity, yshift=-0.03cm] (0,-\height)
                ellipse [x radius = \xradius+\i/40, y radius = \yradius+\i/20/\factor];
              \path pic {disc};
            }
          },%
        }
        % proccess images
            \begin{scope}[xshift = 1cm, yshift = 1cm]
                \shade[top color = gray!50, bottom color = gray!80] (0,0) rectangle (2.5,6);
                \draw (0,0) rectangle (2.5,6);
                \shade[top color = gray!50, bottom color = gray!80] (2.5,0) rectangle (3,6);
                \draw (2.5,0) rectangle (3,6);
                \draw[left color=copper!80, right color=copper!40]
                (3.4,6) rectangle (3.7,8);
                \draw[left color=copper!80, right color=copper!40]
                (3.7,6) rectangle (3.8,8);
                \draw[left color=copper!80, right color=copper!40]
                (1.4,6)--(1.4,7.1)--(3.55,7.1)--(3.55,7)--(1.5,7)--(1.5,6)--cycle;
            \end{scope}
            \begin{scope}[xshift = 6cm, yshift = 1cm]
                \clip (1.1,1) arc(-90:90:0.5 and 2) -- (1,5) arc(90:270:0.5 and 2) -- (1.1,1);
                \path[transform shape, rotate = 90] (3,-1) pic {disc bottom};
                \begin{pgfonlayer}{bg}
                  \shade[top color = gray!20, bottom color = gray!40] (0,0) rectangle (2.5,6);
                  \draw (0,0) rectangle (2.5,6);
                  \shade[left color = gray!20, right color = gray!40] (2.5,0) rectangle (3,6);
                  \draw (2.5,0) rectangle (3,6);
                  \shade[left color = copper!50, right color = copper!90] (3,0) rectangle (3.5,6);
                  \draw (3,0) rectangle (3.5,6);
                  \draw[left color=copper!80, right color=copper!40]
                  (1.9,6)--(2,6)--(2,7.1)--(-0.4,7.1)--(-0.4,7)--(1.9,7)--cycle;
                  \draw[left color=copper!80, right color=copper!40]
                  (-0.4,6.5) rectangle (-0.5,7.5);
                  \draw[left color=copper!80, right color=copper!40]
                  (-0.8,6.5) rectangle (-0.5,7.5);
                  %\draw (-0.6,7.5) rectangle (-0.8,6.5);
                \end{pgfonlayer}
            \end{scope}
            \fill[blue!40, opacity = 0.3] (0,0) rectangle (10.5,6.5);
            \draw[line width = 2mm, color = gray] (0,7) -- (0,0) -- (10.5,0) -- (10.5,7);
            \node at (5,9.3) {$U_\mathrm{anodizing}=\SI{40}{\volt}$};
            \draw (0.5,6) -- (0.5,7.5) node[anchor = south] {Oxalic acid};
            \draw (2,6.7) -- (2,8.5) node[anchor = south] {Platinum};
            \draw (9.25,6.7) -- (9.25,8.5) node[anchor = south] {Copper};
            \draw (9.2,8.5) -- (8,8);
            \draw (6.3,6.8) -- (5.8,6.8) node[anchor = east] {Insulation};
            \draw (7,5.5) -- (7,7.3) node[anchor = south] {wafer};
    \end{tikzpicture}
    \caption{Setup for the anodizing of the membrane production. The aluminum wafer functions as anode while a platinum plate makes for the  cathode. The oxalic acid is stired during the whole process. }
    \label{fig:anodizing}
\end{figure}

\end{document}
