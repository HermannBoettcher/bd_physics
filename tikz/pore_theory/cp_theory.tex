\documentclass[../../thesis.tex]{subfiles}

\begin{document}
\begin{figure}[p]
    \centering
    \subfloat[]{
    \tikzsetnextfilename{closed_straight_pore_1}
        \begin{tikzpicture}[every node/.style = {font = \small},
                        pore/.style = {line width = 0.1cm, color = gray},
                        film/.style = {line width = 0.1cm, color = blue!50},
                        thick_film/.style = {line width = 0.3cm, color = blue!50},
                        graph_label/.style={color = gray, line width = 0.05cm},
                        pore_collapse/.style = {color = blue!50, line width = 0.07cm, ->},
                        MyArrow/.style={single arrow, draw, minimum width=8mm, minimum height=3mm,
                        inner sep=0mm, single arrow head extend=1mm},
                        fontscale/.style = {font=\relsize{#1}},
                        arrowstyle/.style = {scale=2},
                        MyArrow/.style={single arrow, draw, minimum width=8mm, minimum height=5mm, inner sep=0mm, single arrow head extend=1mm},
                        directed/.style={postaction={decorate,decoration={markings, mark=at position .65 with {\arrow[arrowstyle]{stealth}}}}}]
            \pgfdeclarelayer{bg}    % declare background layer
            \pgfdeclarelayer{bbg}    % declare backbackground layer
            \pgfsetlayers{bbg,bg,main}  % set the order of the layers (main is the standard layer)
            \begin{scope}
                \foreach \Xcoor in {0, 2.8, 6.3, 9.1, 12.6}
                \draw[pore] (\Xcoor,3.5) -- (\Xcoor,0) -- (\Xcoor + 1.4,0) -- (\Xcoor + 1.4,3.5);
                \draw (0.4,3.65) -- (0,3.65);
                \draw (0.9,3.65) -- (1.4,3.65);
                \node at (0.7,3.65) {$d$};
                \node[draw = orange!70, fill = orange!50, MyArrow] at (2.15, 2.5) {\phantom{arrow}};
                \node[draw = orange!70, fill = orange!50, MyArrow, rotate=180] at (2.15, 1.5) {\phantom{arrow}};
                \node[draw = green!70, fill = green!50, MyArrow] at (4.85, 2.5) {\phantom{arrow}};
                \node[draw = green!70, fill = green!50, MyArrow, rotate=180] at (4.85, 1.5) {\phantom{arrow}};
                \node[draw = red!70, fill = red!50, MyArrow] at (8.35, 2.5) {\phantom{arrow}};
                \node[draw = red!70, fill = red!50, MyArrow,rotate=180] at (8.35, 1.5) {\phantom{arrow}};
                \node[draw = blue!70, fill = blue!50, MyArrow] at (11.85, 2.5) {\phantom{arrow}};
                \node[draw = blue!70, fill = blue!50, MyArrow,rotate=180] at (11.85, 1.5) {\phantom{arrow}};
                %
                \draw[color = red!60, line width = 0.05cm] (5.6,-0.2) -- (11.2,-0.2) -- (11.2,3.7) -- (5.6,3.7) -- cycle;
                \node at (8.4,-0.6) {Equilibrium pressure};
                \node at (3.5,-0.4) {Spherical meniscus forming};
                \begin{pgfonlayer}{bg}    % select the background layer
                    %filmed pore
                    \draw[film] (2.88,0) -- (2.88,3.5);
                    \draw[film] (4.12,0) -- (4.12,3.5);
                    \draw[film] (2.8,0.08) -- (4.2,0.08);
                    \fill[blue!50] (6.3,0) -- (6.3,3.5) -- (6.5,3.5) -- (6.5,0.7) -- (6.5,0.7) arc[start angle = -180, end angle = 0, radius = 0.5] -- (7.5,3.5) -- (7.7,3.5) -- (7.7,0) -- cycle;
                    \draw[color = blue!50, ->, line width = 0.5mm] (7,0.2) -- (7,3.3);
                    %full pore with menisci
                    \fill[blue!50] (9.1,0) -- (9.1,3.5)  -- (10.5,3.5) -- (10.5,0) -- cycle ;
                    \path [draw = none, fill = white](9.3,3.6) arc[start angle = -180, end angle = 0, radius=0.5];
                    \draw[color = white, ->, line width = 0.5mm] (9.8,3.3) -- (9.8,0.4);
                    %full pore
                    \fill[blue!50] (12.6,0) -- (12.6,3.5)  -- (14,3.5) -- (14,0) -- cycle ;
                \end{pgfonlayer}
            \end{scope}
            \begin{scope}[xshift=1cm,yshift=5cm]
                \draw (0,0) -- (0,3.5);
                \draw (12,0) -- (12,3.5);
                \node[rotate=90] at (-0.7,1.75) {Liquid fraction};
                \draw (0,0) -- (12,0);
                \draw (0,3.5) -- (12,3.5);
                \node at (6,-0.7) {Pressure};
                %yellow
                \draw[color = orange, directed] (0,0) -- (4,0.2);
                \draw[color = orange, directed] (4,0.2)--(0,0);
                %green
                \draw[color = green, directed] (4,0.2) -- (8,0.4);
                \draw[color = green, directed] (8,0.4)--(4,0.2);
                %red
                \draw[color = red, directed] (8,0.4) -- (8,3.35);
                \draw[color = red, directed] (8,3.35)--(8,0.4);
                %blue
                \draw[color = blue, directed] (11,3.5)--(8,3.35);
                \draw[color = blue, directed] (8,3.35)--(11,3.5);
                %DASHED
                \draw[dashed] (11,0) -- (11,3.5);  %psat
                \draw[dashed] (9,0) -- (9,0.15);  %psp
                \draw[dashed] (8,0) -- (8,0.3);  %peq
                %labels
                \node at (-0.2,-0.3) {$0$};
                \node at (-0.2,3.5) {$1$};
                \node at (11,-0.3) {$P_\mathrm{sv}$};
                \node at (8,-0.3) {$P_\mathrm{eq}$};
                \node at (9,-0.3) {$P_\mathrm{sp}$};
            \end{scope}
        \end{tikzpicture}
        \label{fig:theory-cp}
        } \\
        \subfloat[]{
        \tikzsetnextfilename{closed_funneled_pore_3}
        \begin{tikzpicture}[every node/.style = {font = \small},
                            pore/.style = {line width = 0.1cm, color = gray},
                            film/.style = {line width = 0.1cm, color = blue!50},
                            thick_film/.style = {line width = 0.3cm, color = blue!50},
                            graph_label/.style={color = gray, line width = 0.05cm},
                            pore_collapse/.style = {color = blue!50, line width = 0.07cm, ->},
                            MyArrow/.style={single arrow, draw, minimum width=8mm, minimum height=3mm,
                            inner sep=0mm, single arrow head extend=1mm}]
            \tikzset{fontscale/.style = {font=\relsize{#1}}}
            \tikzset{MyArrow/.style={single arrow, draw, minimum width=8mm, minimum height=5mm, inner sep=0mm, single arrow head extend=1mm}}
            \pgfdeclarelayer{bg}    % declare background layer
            \pgfdeclarelayer{bbg}    % declare backbackground layer
            \pgfsetlayers{bbg,bg,main}  % set the order of the layers (main is the standard layer)
            \tikzstyle arrowstyle=[scale=2]
            \tikzstyle directed=[postaction={decorate,decoration={markings,
                mark=at position .65 with {\arrow[arrowstyle]{stealth}}}}]
            \begin{scope}
                \foreach \Xcoor in {0, 2.8, 6.3, 9.1, 12.6}
                \draw[pore] (\Xcoor,3.5) -- (\Xcoor + 0.2,0) -- (\Xcoor + 1.4 - 0.2,0) -- (\Xcoor + 1.4,3.5);
                \draw (0.4,3.65) -- (0,3.65);
                \draw (0.9,3.65) -- (1.4,3.65);
                \node at (0.7,3.65) {$d$};
                \draw (0.4,-0.3) -- (0 + 0.2,-0.3);
                \draw (1,-0.3) -- (1.4 - 0.2,-0.3);
                \node at (0.7,-0.3) {$d'$};
                \node[draw = orange!70, fill = orange!50, MyArrow] at (2.1, 2.5) {\phantom{arrow}};
                \node[draw = orange!70, fill = orange!50, MyArrow,rotate=180] at (2.1, 1.5) {\phantom{arrow}};
                \node[draw = green!70, fill = green!50, MyArrow] at (4.9, 2.5) {\phantom{arrow}};
                \node[draw = green!70, fill = green!50, MyArrow,rotate=180] at (4.9, 1.5) {\phantom{arrow}};
                \node[draw = red!70, fill = red!50, MyArrow] at (8.4, 2.5) {\phantom{arrow}};
                \node[draw = red!70, fill = red!50, MyArrow, rotate=180] at (8.4, 1.5) {\phantom{arrow}};
                \node[draw = blue!70, fill = blue!50, MyArrow] at (11.9, 1.5) {\phantom{arrow}};
                \node[draw = blue!70, fill = blue!50, MyArrow, rotate=180] at (11.9, 1.5) {\phantom{arrow}};
                %
                \draw[color = red!60, line width = 0.05cm] (5.6,-0.2) -- (11.2,-0.2) -- (11.2,3.7) -- (5.6,3.7) -- cycle;
                \node at (8.4,-0.6) {Equilibrium pressure};
                \node at (3.5,-0.4) {Spherical meniscus forming};
                \begin{pgfonlayer}{bg}    % select the background layer
                    %filmed pore
                    \draw[film] (2.88 + 0.2,0) -- (2.88,3.5);
                    \draw[film] (4.12 - 0.2,0) -- (4.12,3.5);
                    \draw[film] (2.8 + 0.2,0.08) -- (4.2 - 0.2,0.08);
                    \fill[blue!50] (6.3 + 0.2,0) -- (6.3,3.5) -- (6.5,3.5) -- (6.5 + 0.16,0.7) -- (6.5 + 0.16,0.7) arc[start angle = -180, end angle = 0, radius = 0.5 - 0.16] -- (7.5,3.5) -- (7.7,3.5) -- (7.7 - 0.2,0) -- cycle;
                    \draw[color = blue!50, ->, line width = 0.5mm] (7,0.2) -- (7,3.3);
                    %full pore with menisci
                    \fill[blue!50] (9.1 + 0.2,0) -- (9.1,3.5)  -- (10.5,3.5) -- (10.5 - 0.2,0) -- cycle ;
                    \path [draw = none, fill = white](9.3,3.6) arc[start angle = -180, end angle = 0, radius=0.3];
                    \draw[color = white, ->, line width = 0.5mm] (9.8,3.3) -- (9.8,0.4);
                    %full pore
                    \fill[blue!50] (12.6 + 0.2,0) -- (12.6,3.5)  -- (14,3.5) -- (14 - 0.2,0) -- cycle ;
                \end{pgfonlayer}
            \end{scope}
            \begin{scope}[xshift=1cm,yshift=5cm]
                \draw (0,0) -- (0,3.5);
                \draw (12,0) -- (12,3.5);
                \node[rotate=90] at (-0.7,1.75) {Liquid fraction};
                \draw (0,0) -- (12,0);
                \draw (0,3.5) -- (12,3.5);
                \node at (6,-0.7) {Pressure};
                %yellow
                \draw[color = orange, directed] (0,0) -- (4,0.2);
                \draw[color = orange, directed] (4,0.2)--(0,0);
                %green
                \draw[color = green, directed] (4,0.2) -- (6.5,0.275);
                \draw[color = green, directed] (6.5,0.275)--(4,0.2);
                %red
                \draw[color = red, directed] (6.5,0.275) -- (8,3.35);
                \draw[color = red, directed] (8,3.35)--(6.5,0.275);
                %blue
                \draw[color = blue, directed] (8,3.35) -- (11,3.5);
                \draw[color = blue, directed] (11,3.5)--(8,3.35);
                %DASHED
                \draw[dashed] (11,0) -- (11,3.5);  %psat
                \draw[dashed] (9,0) -- (9,0.15);  %psp
                \draw[dashed] (8,0) -- (8,0.15);  %peq
                \draw[dashed] (8,3.5) -- (8,3.35);  %psp
                \draw[dashed] (6.5,3.5) -- (6.5,0.35);  %peq
                %labels
                \node at (-0.2,-0.3) {$0$};
                \node at (-0.2,3.5) {$1$};
                \node at (11,-0.3) {$P_\mathrm{sv}$};
                \node at (8,-0.3) {$P_\mathrm{eq}(d)$};
                \node at (9,-0.3) {$P_\mathrm{sp}(d)$};
                \node at (6.5,3.8) {$P_\mathrm{eq}(d')$};
                \node at (8,3.8) {$P_\mathrm{sp}(d')$};
            \end{scope}
        \end{tikzpicture}
        \label{fig:theory-funnelled-cp}} \\
    \caption{Condensation and evaporation isotherm model of a closed cylindrical pore. \protect\subref{fig:theory-cp} corresponds to the mechanisms in a straight pore whereas \protect\subref{fig:theory-funnelled-cp} does for a funnelled one. The corresponding processes inside the pore are illustrated below the isotherm itself. Colors of the arrows between pore states and the respecting pressure range of the graph match.}
    \label{fig:closed-straight-cyl-pore-isos}
\end{figure}
\end{document}
